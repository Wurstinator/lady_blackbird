\ClearShipoutPicture 
\AddToShipoutPicture{%
 	\AtPageLowerLeft{\includegraphics[width=\paperwidth,height=\paperheight,page=3]{../notext_lb.pdf}}
}
	
	% Header
	\placetextbox{0}{0.935}{\paperwidth}{\centering \addfontfeature{LetterSpace=0}
{\fontsize{30}{1}\scshape Natasha Syri}
\textrm{\fontsize{26}{1} \ding{66}}
{\fontsize{30}{1}\scshape Lady Blackbird}}
	\placetextbox{0}{0.9}{\paperwidth}{\fontsize{14}{1}\centering\itshape\addfontfeature{LetterSpace=0} An Imperial noble, in disguise, escaping an arranged marriage so she can be with her lover}
	
	% Traits
	\placetextbox{0.089}{0.86}{0.4\paperwidth}{\fontsize{13}{0}\fontcelestiaantiqua\scshape TRAITS}
	\placetextbox{0.089}{0.83}{0.4\paperwidth}{\fontsize{12}{0}\fontcelestiaantiqua
Imperial Noble \vspace{28pt} \\
Master Sorcerer \vspace{28pt} \\
Athletic \vspace{18pt} \\
Charm \vspace{18pt} \\
Cunning}
	\placetextbox{0.12}{0.813}{0.37\paperwidth}{\fontsize{10}{0}\fontcelestiaantiqua\itshape
Etiquette, Dance, Educated, History, Science, Wealth, Connections, House Blackbird \vspace{17pt} \\
Spellcaster, Channeling, Stormblood, Wind, Lightning, [Fly], [Blast], [Sense] \vspace{17pt} \\
Run, Fencing, Rapier, Duels, Shooting, [Pistol], [Acrobatics] \vspace{17pt} \\
Charisma, Presence, Command, Nobles, Servants, [Soldiers] \vspace{17pt} \\
Deception, Misdirection, Disguise, Codes, [Sneak], [Hide] }

	% Keys and Secrets
	\placetextbox{0.515}{0.86}{0.4\paperwidth}{\fontsize{12}{0}\fontcelestiaantiqua
Key of the Paragon \vspace{48pt} \\
Key of the Mission \vspace{54pt} \\
Key of the Impostor \vspace{48pt} \\
Secret of Stormblood \vspace{38pt} \\
Secret of Inner Focus}
	\placetextbox{0.545}{0.843}{0.37\paperwidth}{\fontsize{10}{0}\fontcelestiaantiqua
\textit{As a noble, you’re a cut above the common man. Hit your key
when you demonstrate your superiority or when your noble traits
overcome a problem.} Buyoff: Disown your noble heritage. \vspace{17pt} \\
\textit{You must escape the Empire and rendezvous with your once secret
lover, the Pirate King Uriah Flint, whom you haven’t seen in six
years. Hit your key when you take action to complete the mission.}
Buyoff: Give up on your mission. \vspace{17pt} \\
\textit{You are in disguise, passing yourself off as commoner. Hit your key
when you perform well enough to fool someone with your disguise.}
Buyoff: Reveal your true identity to someone you fooled. \vspace{17pt} \\
\textit{As long as you can speak, you can channel magical power and do
Sorcery. You have the Master Sorcerer trait and the Stormblood tag.} \vspace{17pt} \\
\textit{Once per session, you can re-roll a failure when doing Sorcery.} }
	
	% Rules Summary 1
	\placetextbox{0.089}{0.435}{0.4\paperwidth}{\fontsize{8}{0}\fontcelestiaantiqua 
{\fontsize{14}{0}\fontcelestiaantiqua\scshape Rolling the Dice} \vspace{3pt} \\
When you try to overcome an obstacle, you roll dice. Start with one die.
Add a die if you have a \textbf{trait} that can help you. If that trait has any \textbf{tags}
that apply, add another die for each tag. Finally, add any number of dice
from your personal \textbf{pool} of dice (your pool starts with 7 dice). \vspace{5pt} \\
Roll all the dice you’ve gathered. Each die that shows \textbf{4 or higher} is a hit.
You need hits equal to the difficulty \textbf{level} (usually 3) to pass the obstacle. \vspace{5pt} \\
	\textsc{levels: 2 easy—3 difficult—4 challenging—5 extreme} \vspace{5pt} \\
	\textbf{If you pass,} discard all the dice you rolled (including any pool dice you
used). Don’t worry, you can get your pool dice back. \vspace{5pt} \\
	\textbf{If you don't pass,}  you don’t yet achieve your goal. But, you get to keep
the pool dice you rolled and \textbf{add another die to your pool}. The GM will
escalate the situation in some way and you might be able to try again. \vspace{6pt} \\
{\fontsize{14}{0}\fontcelestiaantiqua\scshape Conditions} \vspace{3pt} \\
	When events warrant or especially when you fail a roll, the GM may
impose a \textbf{condition} on your character: \textbf{Injured, Dead, Tired, Angry,
Lost, Hunted,} or \textbf{Trapped}. When you take a condition, mark its box
and say how it comes about. [\textsc{Note}: The “dead” condition just means
“presumed dead” unless you say otherwise.] \vspace{6pt} \\
{\fontsize{14}{0}\fontcelestiaantiqua\scshape Helping} \vspace{3pt} \\
	If your character is in a position to help another character, you can give them
a die from your pool. Say what your character does to help. If the roll
fails, you get your pool die back. If it succeeds, your die is lost.
	}
	
	% Rules Summary 2
	\placetextbox{0.515}{0.435}{0.4\paperwidth}{\fontsize{8}{0}\fontcelestiaantiqua 
{\fontsize{14}{0}\fontcelestiaantiqua\scshape Keys} \vspace{3pt} \\
When you hit a Key, you can do one of two things:
\begin{itemize}
	\setlength\itemsep{0em}
	\item Take an \textbf{experience point} (XP)
	\item Add a die to your pool (up to a max of 10)
\end{itemize}
	If you go into danger because of your key, you get 2 XP or 2 pool dice
(or 1 XP and 1 pool die). When you have accumulated 5 XP, you earn an
\textbf{advance}. You can spend an advance on one of the following:
\begin{itemize}
	\setlength\itemsep{0em}
	\item Add a new \textbf{Trait} (based on something you learned during play or
on some past experience that has come to light)
	\item Add a \textbf{tag} to an existing trait
	\item Add a new \textbf{Key} (you can never have the same key twice)
	\item Learn a \textbf{Secret} (if you have the means to)
\end{itemize}
	You can hold on to advances if you want, and spend them at any time,
even in the middle of a battle! \vspace{5pt} \\
	Each key also has a \textbf{buyoff}. If the buyoff condition occurs, you have the
option of removing the Key and earning two advances. \vspace{6pt} \\
{\fontsize{14}{0}\fontcelestiaantiqua\scshape Refresh} \vspace{3pt} \\
	You can refresh your pool back to 7 dice by having a \textbf{refreshment scene}
with another character. You may also remove a condition or regain the
use of a Secret, depending on the details of the scene. A refreshment
scene is a good time to ask questions (in character) so the other player
can show off aspects of his or her PC—“Why did you choose this
life?”—“What do you think of the Lady?”—“Why did you take this
job?” etc. Refreshment scenes can be flashbacks, too.
	}
	
	
	\placetextbox{0.089}{0.55}{0.4\paperwidth}{\fontsize{9}{0}\fontcelestiaantiqua  \addfontfeature{LetterSpace=0}
Tags in [brackets] are qualities you don’t have yet. You can buy them with
advances. See the Rules Summary below, under Keys.}
	\placetextbox{0}{0.465}{\paperwidth}{\centering
\textrm{\fontsize{20}{1} \ding{66}}
{\fontsize{24}{1}\scshape rules summary}
\textrm{\fontsize{20}{1} \ding{66}}
}

	
\placetextbox{0.155}{0.506}{0.1\paperwidth}{\fontsize{12}{0}\fontcelestiaantiqua\scshape injured}
\placetextbox{0.281}{0.508}{0.1\paperwidth}{\fontsize{12}{0}\fontcelestiaantiqua\scshape dead}
\placetextbox{0.376}{0.508}{0.1\paperwidth}{\fontsize{12}{0}\fontcelestiaantiqua\scshape tired}
\placetextbox{0.478}{0.508}{0.1\paperwidth}{\fontsize{12}{0}\fontcelestiaantiqua\scshape angry}
\placetextbox{0.587}{0.508}{0.1\paperwidth}{\fontsize{12}{0}\fontcelestiaantiqua\scshape lost}
\placetextbox{0.681}{0.508}{0.1\paperwidth}{\fontsize{12}{0}\fontcelestiaantiqua\scshape hunted}
\placetextbox{0.802}{0.508}{0.1\paperwidth}{\fontsize{12}{0}\fontcelestiaantiqua\scshape trapped}