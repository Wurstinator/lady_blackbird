\phantom{.}
	
\ClearShipoutPicture 
\AddToShipoutPicture{%
 	\AtPageLowerLeft{\includegraphics[width=\paperwidth,height=\paperheight,page=9]{../notext_lb.pdf}}
}

	% Header
	\placetextbox{0}{0.935}{\paperwidth}{\centering \addfontfeature{LetterSpace=0}
\textrm{\fontsize{26}{1} \ding{66}}
{\fontsize{30}{1}\scshape Ein Spiel leiten}
\textrm{\fontsize{26}{1} \ding{66}}}
	\placetextbox{0}{0.9}{\paperwidth}{\fontsize{14}{1}\centering\addfontfeature{LetterSpace=0}\itshape Hinweise und Hilfen für den Spielleiter}
		\placetextbox{0}{0.465}{\paperwidth}{\centering
\textrm{\fontsize{20}{1} \ding{66}}
{\fontsize{24}{1}\scshape hindernisse \& probleme}
\textrm{\fontsize{20}{1} \ding{66}}}
	
	% Top half
	\placetextbox{0.089}{0.87}{0.4\paperwidth}{\fontsize{8}{10}\fontcelestiaantiqua
{\fontsize{12}{0}\fontcelestiaantiqua\scshape Hör zu \& stell Fragen, anstatt zu planen} \\
Als Spielleiter solltest du nicht versuchen, vorauszuplanen was als nächstes geschieht. Stell stattdessen Fragen—möglichst viele Fragen und fokussiere sie auf Dinge von Interesse. Zum Beispiel könnte Cyrus Naomi einen Befehl geben, während Lady Blackbird dabei steht, aber der Spieler der Lady bemerkt dies nicht sofort. Naomi fängt nun an, dem Befehl zu folgen. Also kann der SL den Spieler von Lady Blackbird fragen: \enquote{Wie findest du, dass der Captain deine Leibwache herumschickt? Ist das okay?} Und wenn Lady Blackbird das überhaupt nicht okay findet, kannst du weiter fragen \enquote{Was sagst du zu ihm? Was sagst du zu Naomi?} Ein paar mehr solcher Fragen und die Spieler liegen sich in den Haaren und wollen ihren Willen mit den Würfeln durchsetzen. \vspace{5pt} \\
Weitere mögliche Fragen sind: \vspace{2pt} \\
\phantom{..} “Geht was kaputt, während du dieses abgedrehte Manöver durchführst?”  \\
\phantom{..} “Das Feuer breitet sich immer schneller aus, oder?”  \\
\phantom{..} “Das hört sich nach einem gewagten Plan an. Was ist der erste Schritt?” \\
\phantom{..} “Seid ihr an einen einsamen Ort? Passiert etwas zwischen euch?”  \\
\phantom{..} “Was weißt du über den Widerstand \enquote{Purpurwolken}? Was sind das für Leute? Ist es normal, dass sie sich so weit in imperiale Gebiete bewegen?”  \vspace{5pt} \\
Du kannst das Spiel gut voranbringen, wenn du durchgehend derartige Fragen stellst. Teil deiner Aufgabe ist es, den Spielern aufmerksam zuzuhören und es von allen Seiten zu betrachten, ob du damit etwas ungewolltes verursachen kannst. \vspace{5pt} \\
\textbf{Die Aufgaben des SL}: zuhören und aufgreifen; die NPCs mit Begeisterung spielen; interessante Hindernisse entwickeln und ggf.\ Zustände auf die Spieler verteilen (vor allem wenn Würfe fehlschlagen).
}


\placetextbox{0.50}{0.87}{0.42\paperwidth}{\fontsize{8}{5}\fontcelestiaantiqua
{\fontsize{12}{0}\fontcelestiaantiqua\scshape Stimme zu \& Erzeuge Hindernissen} \\
Standardmäßig können Charaktere alles erfolgreich tun, was von ihren Eigenschaften abgedeckt wird. In anderen Worten, sie sind kompetente und effektiv. Es macht keinen Spaß, wenn die Spieler jedes Mal würfeln müssen, um etwas zu tun, obwohl kein cooles Hindernis im Weg steht. Stimme einfach der Beschreibung der Action zu und stell die üblichen Fragen. Sie aber auch immer aufmerksam auf Gelegenheiten, Hindernisse in die sich entwickelnden Ereignisse einzubauen. Solange du deine Fragen stellst und gut zuhörst, sollten sich diese Gelegenheiten dauerhaft ergeben. \vspace{2pt} \\
Hindernisse können viele Formen annehmen: Leute (Piraten, Goblins, Imperiale, einfache Bürger, Adelige, \dots), Wetter, Monster (Sternenkalmare, Flugaale, \dots), Situationen (Feuer, Herunterfallen, beschossen werden, Verfolgungen, Flüchte, \dots) oder beliebige andere Schwierigkeiten, die dir einfallen. \vspace{2pt} \\
Wenn ein PC etwas versucht, was nicht von den Eigenschaften abgedeckt wird, ist das schon ein Hindernis, nämlich Mangel an Erfahrung und Übung. Viele Dinge können schief gehen, wenn du keine Ahnung von dem hast, was du gerade tust! Spieler können auch bewusst Dinge probieren, in denen ihre Charaktere schlecht sind, um ihren Würfelpool aufzustocken. In dem Fall kannst du die Geschichte weiter eskalieren lassen und jeder hat etwas davon. \vspace{10pt} \\
{\fontsize{12}{0}\fontcelestiaantiqua\scshape Zustände} \\
Ein Zustand schränkt ein, was ein Spieler über seinen Charakter erzählen kann. Es ist ein Hinweis, dass der SL und die Spieler besondere Aufmerksamkeit auf diesen Zustand lenken sollten und es als zusätzliches Material zur Weiterentwicklung der Story nutzen können. Das Spiel besteht prinzipiell nur daraus, dass wir miteinader reden. Wenn ein Spieler also überlegt \enquote{Was könnte ich jetzt sagen?} und dann bemerkt, dass er den Zustand \enquote{Wütend} hat, fängt er an mit \enquote{Oh, ich bin richtig wütend auf jemanden. Snargle! Wieso sind wir noch nicht in Nachtdock, du Nichtsnutz?!} Für den Spielleiter können die Zustände zusätzliche Gelegenheiten für Hindenisse o.ä.\ erschaffen: \enquote{Du bist verletzt, korrekt? Die Riesenspinnen können Blut riechen. Sie ignorieren alle anderen und stürmen auf dich zu.} \\
Manchmal kann ein Zustand auch ein Hindernis an sich sein und \\
einen Würfelwurf erfordern, um sich damit zu beschäftigen.
}
	
	
	% Bottom half
	
	\placetextbox{0.09}{0.44}{0.3\paperwidth}{\fontsize{12}{0}\fontcelestiaantiqua Flucht aus der Brigg}
	\placetextbox{0.09}{0.23}{0.3\paperwidth}{\fontsize{12}{0}\fontcelestiaantiqua Hinterhalt von Kopfgeldjägern}
	\placetextbox{0.37}{0.44}{0.3\paperwidth}{\fontsize{12}{0}\fontcelestiaantiqua Luftschiff-Schlacht}
	\placetextbox{0.37}{0.235}{0.3\paperwidth}{\fontsize{12}{0}\fontcelestiaantiqua Verhandeln mit Schurken}
	\placetextbox{0.65}{0.44}{0.3\paperwidth}{\fontsize{12}{0}\fontcelestiaantiqua Angriff der Sternenkalmare}
	\placetextbox{0.65}{0.235}{0.3\paperwidth}{\fontsize{12}{0}\fontcelestiaantiqua Kampf gegen einen Beschwörer}
	
	
	\placetextbox{0.10}{0.42}{0.26\paperwidth}{\fontsize{9}{0}\fontcelestiaantiqua 
\textit{Die Zellen in der Brig der Gramhand sind mit Stahlwänden und schweren Eisenschlössern an den Türen ausgestattet.} \vspace{5pt} \\
\textsc{Hindernisse}:
Schloss knacken: 3. 
Einen Wächter austricksen: 3.
(Nur Naomi Bishop—Tür aufbrechen: 0. Leise aufbrechen: 5)
Durch das Schiff schleichen: 4.
Kampf gegen Besatzung: 3.
Kampf gegen Marines: 4.
Kampf gegen viele Marines: 5+. \vspace{5pt} \\
\textsc{Eskalation}: 
Alarm wird ausgelöst.
Mehr Marines kommen.
Die \enquote{Eule} wird von der Gramhand getrennt, um die Flucht zu verhindern.
Jemand wird von der Gruppe getrennt (Veirrt / Gefangen).}
	
	\placetextbox{0.10}{0.21}{0.26\paperwidth}{\fontsize{9}{0}\fontcelestiaantiqua 
\textit{Solange sie nicht untertauchen, werden die Handlungen der \enquote{Eule} irgendwann die Aufmerksamkeit von Kopfgeldjägern anziehen, die auf der Suche nach Vance oder Lady Blackbird sind.} \vspace{5pt} \\
\textsc{Hindernisse}:
Kampf aus dem Hinterhalt: 5.
Fliehen: 3.
Verhandeln: 4.
Durch einen miesen Trick das Blatt wernden: 3. \vspace{5pt} \\
\textsc{Eskalation}:
Jemand wird als Geisel \linebreak
\phantom{....} genommen (Gefangen).
	}
	
	\placetextbox{0.38}{0.42}{0.26\paperwidth}{\fontsize{9}{0}\fontcelestiaantiqua 
\textit{In einem Kampf der Lufschiffe willst du dich immer über deinem Gegner befinden—es sei denn, dein Schiff ist für die niederen Tiefen ausgerüstet \dots} \vspace{1pt} \\
\textsc{Hindernisse}:
Für einen freien Schuss manövrieren: 3.
Gegen ein kleineres, schnelleres Schiff lenken: 4.
Entern: 4.
Auf anderes Schiff schießen: 3.
Gegen ein kleineres Schiff: 4.
Feuer ausweichen: 3.
Mehreren Angriffen ausweichen: 4-5. \vspace{1pt} \\
\textsc{Eskalation}: 
Die \textit{Eule} wurde getroffen und ist nicht mehr zu steuern. (Angeschlagen / Langsam).
Weitere feindliche Schiffe tauchen auf.
Du wirst in einen Sturm getrieben.
Der Kampf lockt einen Sternenkalmar herbei.}
	
	\placetextbox{0.38}{0.215}{0.26\paperwidth}{\fontsize{9}{0}\fontcelestiaantiqua 
\textit{Um den geheimen Weg zur Festung des Piratenkönigs in den Verbliebenen zu finden, muss man mit einer Menge widerwärtigen Leute arbeiten.} \vspace{1pt} \\
\textsc{Hindernisse}:
Ein Schlupfloch der Unterwelt finden: 3.
Man will sich nicht mit dir anlegen: 3.
Ein faires Geschäft machen: 4.
Ein Geschäft zu deinen Gunsten machen: 5.
Lügen durchschauen: 4. \vspace{1pt} \\
\textsc{Eskalation}: 
Die Halunken entschließen sich, einfach von dir zu nehmen, was sie haben wollen.
Du wirst gegen Geld verraten.
Dir wurde zum Treffpunkt gefolgt.}
	
	\placetextbox{0.66}{0.42}{0.26\paperwidth}{\fontsize{9}{0}\fontcelestiaantiqua 
\textit{Während man durch die niederen Tiefen fliegt, können Motoren und Antriebe hungrige Sternenkalmare anlocken. Die Tentakeln wickeln sich langsam um das Schiff \dots} \vspace{5pt} \\
\textsc{Hindernisse}:
Von Tentakeln befreien: 5.
Kalmar angreifen: 3.
Durch Tinte fliegen: 4.
Kalmar-Angriff ausweichen (zerquetschen, beißen, donnernder Gesang): 3. \vspace{5pt} \\
\textsc{Eskalation}: 
Der Kalmar ruft weitere Kalmare mit seinem Gesang.
Kalmarblut lockt andere Monster an.
Weiter in die Tiefen gezogen (Verirrt).
Absturz in Trümmer, Felsen oder versteckte Welten. (Angeschlagen, Zerstört)	}
	
	\placetextbox{0.66}{0.215}{0.26\paperwidth}{\fontsize{9}{0}\fontcelestiaantiqua 
\textit{Uriah Flint ist ein Flammenblut und Meisterbeschwörer. Nicht, dass irgendjemand gegen ihn kämpfen würde. Warum sollte so etwas passieren?} \vspace{5pt} \\
\textsc{Hindernisse}:
Feuerbällen ausweichen: 3.
Durch magische Barrieren angreifen: 5.
Steigende Hitze und Rauch aushalten: 3. \vspace{5pt} \\
\textsc{Eskalation}:
Die Brände wachsen außer Kontrolle.
Die Waffen werden zu heiß, um sie halten zu können.}
	
	
	
	
	
	
	
	
	
	
	
	
	
	
	