

	\placetextbox{0}{0.465}{\paperwidth}{\centering
\textrm{\fontsize{20}{1} \ding{66}}
{\fontsize{24}{1}\scshape regeln kurzform}
\textrm{\fontsize{20}{1} \ding{66}}}

	% Rules Summary 1
	\placetextbox{0.089}{0.438}{0.4\paperwidth}{\fontsize{8}{0}\fontcelestiaantiqua 
{\fontsize{14}{0}\fontcelestiaantiqua\scshape Würfel} \\
Immer wenn du ein Hindernis überwinden willst, wirfst du (sechsseitige) Würfel. Du fängst mit einem Würfel an.
Wenn du eine \textbf{Eigenschaft} hast, die dir bei deinem Versuch helfen kann, füge einen Würfel hinzu.
Falls diese Eigenschaft \textbf{Marker} hat, die auch in dieser Situation zutreffen, kannst du einen weiter
Würfel für jedes dieser Marker nehmen. Zuletzt kannst du noch eine beliebige Anzahl von Würfeln aus deinem persönlichen
\textbf{Würfel-Pool} nehmen (der Pool hat zu Beginn 7 Würfel). Danach wirfst du alle Würfel, die du so gesammelt hast. Jeder
Würfel, der eine \textbf{4 oder höher} zeigt, ist ein Treffer. Du benötigst mindestens so viele Treffer, wie die Schwierigkeit des Hindernisses angibt. \vspace{1pt} \\
	\textsc{level: 2 leicht—3 schwer—4 herausfordernd—5 extrem} \vspace{1pt} \\
	\textbf{Wenn du bestehst,} gib all diese Würfel ab. Keine Angst, du kannst wieder neue Würfel für deinen Pool sammeln.\vspace{1pt} \\
	\textbf{Wenn du nicht bestehst,} hast du das versuchte Ziel nicht erreicht. Aber du kannst die Pool-Würfel aus diesem Wurf behalten und sogar \textbf{einen weiteren hinzufügen} (maximal 10). Der Spielleiter kann nun die Situation eskalieren lassen und du kriegst eventuell eine weitere Chance für einen Versuch. \vspace{3pt} \\
{\fontsize{14}{0}\fontcelestiaantiqua\scshape Zustände} \\
	Wenn ein Ereignis es hergibt oder wenn du einer deiner Würfe fehlschlägt, kann der Spielleiter deinem Charakter einen
	\textbf{Zustand} zuweisen: \textbf{Verletzt, Tot, Wütend, Verirrt, Müde, Verfolgt} oder \textbf{Gefangen}. Wenn du
	von einem Zustand betroffen bist, markierst du das entsprechende Kästchen und erklärst, wie genau es sich an dir auswirkt. [Der “Tot”-Zustand steht für “vermutlich tot” außer du gibst dem Spielleiter dein Einverständnis für anderes.] \vspace{3pt} \\
{\fontsize{14}{0}\fontcelestiaantiqua\scshape Hilfe} \\
	Wenn dein Charakter in der Lage ist, einem anderen zu helfen, kannst du ihm oder ihr einen Würfel aus deinem Pool abgeben und erklären, \linebreak
	\phantom{......} wie dein Charakter dem anderen beisteht. Wenn der Wurf \linebreak
	\phantom{........} fehlschlägt, kriegst du den Würfel zurück. Ansonsten musst du \linebreak
	\phantom{................} ihn abgeben.
	}
	
	% Rules Summary 2
	\placetextbox{0.515}{0.438}{0.4\paperwidth}{\fontsize{8}{-2}\fontcelestiaantiqua 
{\fontsize{14}{0}\fontcelestiaantiqua\scshape Schlüssel} \\
Wenn du einen Schlüssel benutzt, kannst du eine der beiden folgenden Optionen wählen:
\begin{itemize}[topsep=3pt]
	\setlength\itemsep{-3pt}
	\item Erhalte einen \textbf{Erfahrungspunkt} (EP)
	\item Füge deinem Pool einen Würfel hinzu (maximal 10)
\end{itemize}
	Wenn du durch deinen Schlüssel in eine bedrohliche Lage gerätst, kannst du sogar zwei Optionen wählen (oder dieselbe zweimal). Sobald du 5 EP gesammelt hast, kannst du sie gegen einen \textbf{Fortschritt} umtauschen. Ein Umtausch bringt dir einen der folgenden Boni:
\begin{itemize}[topsep=3pt]
	\setlength\itemsep{-3pt}
	\item Lerne eine neue \textbf{Eigenschaft} (basierend auf etwas, was dein Charakter während des Spiels gelernt oder aus seiner Vergangenheit herausgefunden hat).
	\item Füge einer Eigenschaft ein neues \textbf{Marker} hinzu.
	\item Erhalte einen neuen \textbf{Schlüssel}. Du kannst nicht zwei Mal denselben Schlüssel haben.
	\item Lerne ein neues \textbf{Geheimnis}, unter der Bedingung dass du die Voraussetzungen erfüllst.
\end{itemize}
	Du kannst diesen Umtausch jederzeit vornehmen, auch mitten in einem Kampf! 
	Jeder deiner Schlüssel hat außerdem einen \textbf{Austausch}. Wenn die Bedingung des Austausches erfüllt ist, kannst du
	den Schlüssel dauerhaft aufgeben um dafür zwei Fortschritte zu erhalten. \vspace{3pt} \\
{\fontsize{14}{0}\fontcelestiaantiqua\scshape Erholung} \\ 
	Du kannst deinen Würfelpool wieder auf 7 Würfel auffrischen, indem du eine \textbf{Erholungsszene} mit einem anderen Charakter erlebst. Eine solche Szene ist eine gute Zeit, um sich (als Charakter) Fragen zu stellen und dem anderen Spieler die Möglichkeit zu geben, etwas von seiner Geschichte zu erzählen. —“Wieso hast du dich für dieses Leben entschieden?”—“Was denkst du über Natasha/Lady Blackbird?”—“Wieso hast du diesen Auftrag angenommen?” etc. Erfolgsszenen können auch Rückblicke in die Vergangenheit sein.
	Je nach Art der Szene, \\
	können auch Zustände entfernt oder Geheimnise nach einer \\
	Nutzung reaktiviert werden.
	}