\ClearShipoutPicture 
\AddToShipoutPicture{%
 	\AtPageLowerLeft{\includegraphics[width=\paperwidth,height=\paperheight,page=6]{../notext_lb.pdf}}
}
	
	% Header
	\placetextbox{0}{0.935}{\paperwidth}{\centering \addfontfeature{LetterSpace=0}
\textrm{\fontsize{26}{1} \ding{66}}
{\fontsize{30}{1}\scshape Kale Arkam}
\textrm{\fontsize{26}{1} \ding{66}}}
	\placetextbox{0}{0.9}{\paperwidth}{\fontsize{14}{1}\centering\addfontfeature{LetterSpace=0} \textit{Dieb und kleiner Magier. Erster Offizier und Mechaniker der} “Eule”.}
	
	% Traits
	\placetextbox{0.089}{0.86}{0.4\paperwidth}{\fontsize{13}{0}\fontcelestiaantiqua\scshape EIGENSCHAFTEN}
	\placetextbox{0.089}{0.83}{0.4\paperwidth}{\fontsize{12}{0}\fontcelestiaantiqua
Dieb \vspace{26pt} \\
Betrügerisch \vspace{26pt} \\
Kleine Magie {\fontsize{10}{0}\fontcelestiaantiqua (nur ein Marker pro Wurf)} \vspace{26pt} \\
Mechaniker}
	\placetextbox{0.12}{0.813}{0.37\paperwidth}{\fontsize{10}{0}\fontcelestiaantiqua\itshape
Leise, Schleichen, Verstecken, Gewandt, Schlösser, Aufmerksam, Fallen, Dunkelheit, [Alarm], [Ablenkung] \vspace{17pt} \\
Schnell, Unfairer Kampf, Akrobat, Flucht, Schlangenmensch, [Fingerfertigkeit], [Turner], [Messer] \vspace{17pt} \\
Lichtzauber, Dunkelzauber, Sprungzauber, Bruchzauber, [Sammeln], [Magier] \vspace{17pt} \\
Reparieren, Motor, Effizienz, Ersatzteile, Sabotage, [Verbesserungen], [Schiffswaffen] }

	% Keys and Secrets
	\placetextbox{0.515}{0.855}{0.4\paperwidth}{\fontsize{12}{0}\fontcelestiaantiqua
Schlüssel der Gier \vspace{46pt} \\
Schlüssel der Mission \vspace{36pt} \\
Schlüssel der Bruderschaft \vspace{48pt} \\
Geheimnis des Versteckens \vspace{38pt} \\
Geheimnis der Reaktion}
	\placetextbox{0.545}{0.843}{0.37\paperwidth}{\fontsize{10}{0}\fontcelestiaantiqua
\textit{Du magst Dinge die funkeln. Benutze diesen Schlüssel, wenn du etwas cooles stiehlst oder eine große Belohnung erhältst.} Austausch: Entsage dem Leben als Krimineller für immer. \vspace{17pt} \\
\textit{Du musst Lady Blackbird sicher zu ihrem Ziel bringen. Benutze diesen Schlüssel, wenn du aktiv auf dieses Ziel hinarbeitest.}
Austausch: Gib die Mission auf. \vspace{17pt} \\
\textit{Du hast mit Kapitän Vance einen Bruderschwur geleistet. Benutze diesen Schlüssel, wenn du von Vance beeinflusst wirst oder wenn du demonstrierst, wie eng euer Bund ist.}
Austausch: Beende die Beziehung zu Vance. \vspace{17pt} \\
\textit{Egal wie gründlich du durchsucht wirst, du schaffst es immer ein paar essentielle Gegenstände bei dir zu behalten. Du kannst jeden einfachen Gegenstand hervorbringen.} \vspace{17pt} \\
\textit{Einmal pro Sitzung kannst du einen Würfelwurf wiederholen, der mit Eleganz, Geschick oder guten Reflexen zu tun hat.} }
	
	
	

	\placetextbox{0}{0.465}{\paperwidth}{\centering
\textrm{\fontsize{20}{1} \ding{66}}
{\fontsize{24}{1}\scshape rules summary}
\textrm{\fontsize{20}{1} \ding{66}}}

	% Rules Summary 1
	\placetextbox{0.089}{0.435}{0.4\paperwidth}{\fontsize{8}{0}\fontcelestiaantiqua 
{\fontsize{14}{0}\fontcelestiaantiqua\scshape Rolling the Dice} \vspace{3pt} \\
When you try to overcome an obstacle, you roll dice. Start with one die.
Add a die if you have a \textbf{trait} that can help you. If that trait has any \textbf{tags}
that apply, add another die for each tag. Finally, add any number of dice
from your personal \textbf{pool} of dice (your pool starts with 7 dice). \vspace{5pt} \\
Roll all the dice you’ve gathered. Each die that shows \textbf{4 or higher} is a hit.
You need hits equal to the difficulty \textbf{level} (usually 3) to pass the obstacle. \vspace{5pt} \\
	\textsc{levels: 2 easy—3 difficult—4 challenging—5 extreme} \vspace{5pt} \\
	\textbf{If you pass,} discard all the dice you rolled (including any pool dice you
used). Don’t worry, you can get your pool dice back. \vspace{5pt} \\
	\textbf{If you don't pass,}  you don’t yet achieve your goal. But, you get to keep
the pool dice you rolled and \textbf{add another die to your pool}. The GM will
escalate the situation in some way and you might be able to try again. \vspace{6pt} \\
{\fontsize{14}{0}\fontcelestiaantiqua\scshape Conditions} \vspace{3pt} \\
	When events warrant or especially when you fail a roll, the GM may
impose a \textbf{condition} on your character: \textbf{Injured, Dead, Tired, Angry,
Lost, Hunted,} or \textbf{Trapped}. When you take a condition, mark its box
and say how it comes about. [\textsc{Note}: The “dead” condition just means
“presumed dead” unless you say otherwise.] \vspace{6pt} \\
{\fontsize{14}{0}\fontcelestiaantiqua\scshape Helping} \vspace{3pt} \\
	If your character is in a position to help another character, you can give them
a die from your pool. Say what your character does to help. If the roll
fails, you get your pool die back. If it succeeds, your die is lost.
	}
	
	% Rules Summary 2
	\placetextbox{0.515}{0.435}{0.4\paperwidth}{\fontsize{8}{0}\fontcelestiaantiqua 
{\fontsize{14}{0}\fontcelestiaantiqua\scshape Keys} \vspace{3pt} \\
When you hit a Key, you can do one of two things:
\begin{itemize}
	\setlength\itemsep{0em}
	\item Take an \textbf{experience point} (XP)
	\item Add a die to your pool (up to a max of 10)
\end{itemize}
	If you go into danger because of your key, you get 2 XP or 2 pool dice
(or 1 XP and 1 pool die). When you have accumulated 5 XP, you earn an
\textbf{advance}. You can spend an advance on one of the following:
\begin{itemize}
	\setlength\itemsep{0em}
	\item Add a new \textbf{Trait} (based on something you learned during play or
on some past experience that has come to light)
	\item Add a \textbf{tag} to an existing trait
	\item Add a new \textbf{Key} (you can never have the same key twice)
	\item Learn a \textbf{Secret} (if you have the means to)
\end{itemize}
	You can hold on to advances if you want, and spend them at any time,
even in the middle of a battle! \vspace{5pt} \\
	Each key also has a \textbf{buyoff}. If the buyoff condition occurs, you have the
option of removing the Key and earning two advances. \vspace{6pt} \\
{\fontsize{14}{0}\fontcelestiaantiqua\scshape Refresh} \vspace{3pt} \\
	You can refresh your pool back to 7 dice by having a \textbf{refreshment scene}
with another character. You may also remove a condition or regain the
use of a Secret, depending on the details of the scene. A refreshment
scene is a good time to ask questions (in character) so the other player
can show off aspects of his or her PC—“Why did you choose this
life?”—“What do you think of the Lady?”—“Why did you take this
job?” etc. Refreshment scenes can be flashbacks, too.
	}	
	
	\placetextbox{0.089}{0.555}{0.4\paperwidth}{\fontsize{9}{0}\fontcelestiaantiqua  \addfontfeature{LetterSpace=0}
Marker in [Klammern] sind Fähigkeiten, die du noch nicht erlangt hast. Du kannst sie bei einem Fortschritt freischalten. Lies genaueres in der Regelzusammenfassung.}
	
\placetextbox{0.155}{0.506}{0.1\paperwidth}{\fontsize{12}{0}\fontcelestiaantiqua\addfontfeature{LetterSpace=-5}\scshape verletzt}
\placetextbox{0.281}{0.508}{0.1\paperwidth}{\fontsize{12}{0}\fontcelestiaantiqua\addfontfeature{LetterSpace=-5}\scshape tot}
\placetextbox{0.376}{0.508}{0.1\paperwidth}{\fontsize{12}{0}\fontcelestiaantiqua\addfontfeature{LetterSpace=-5}\scshape wütend}
\placetextbox{0.478}{0.508}{0.1\paperwidth}{\fontsize{12}{0}\fontcelestiaantiqua\addfontfeature{LetterSpace=-5}\scshape verirrt}
\placetextbox{0.587}{0.508}{0.1\paperwidth}{\fontsize{12}{0}\fontcelestiaantiqua\addfontfeature{LetterSpace=-5}\scshape müde}
\placetextbox{0.681}{0.508}{0.1\paperwidth}{\fontsize{12}{0}\fontcelestiaantiqua\addfontfeature{LetterSpace=-5}\scshape verfolgt}
\placetextbox{0.802}{0.508}{0.1\paperwidth}{\fontsize{12}{0}\fontcelestiaantiqua\addfontfeature{LetterSpace=-5}\scshape gefangen}