	\phantom{.}
	
	\ClearShipoutPicture 
	\AddToShipoutPicture{%
 		\AtPageLowerLeft{\includegraphics[width=\paperwidth,height=\paperheight,page=2]{../notext_lb.pdf}}
	}
	
	\placetextbox{0}{0.935}{\paperwidth}{\centering
\textrm{\fontsize{35}{1} \ding{66}}
{\fontsize{38}{1} \textsc{ Die Bläue }}
\textrm{\fontsize{35}{1} \ding{66}}}
	\placetextbox{0}{0.89}{\paperwidth}{\fontsize{17}{1}\centering \textit{Zerbrochene Welten, die einen dämmernden Stern umkreisen}}

	\begin{tikzpicture}[remember picture,overlay]
	\path[postaction={decorate},decoration={text along path,
	text={|\addfontfeature{LetterSpace=50}\fontsize{16}{0}\scshape|freie welten},text align=center}]
 	($(current page.center)+(0.05,5.15)$) ++(-84:6.5) arc (-84:90:6.5);	
	
	\path[postaction={decorate},decoration={text along path,
	text={|\fontsize{16}{0}\scshape|imperiale welten},text align=center}]
 	($(current page.center)+(0.05,5.15)$) ++(180:7) arc (180:225:7);	
	
	\path[postaction={decorate},decoration={text along path,
	text={|\fontsize{12}{0}\scshape|imperiale ausdehnung},text align=center}]
 	($(current page.center)+(0.05,5.15)$) ++(180:6) arc (180:135:6);
	\end{tikzpicture}	
	
	\placetextbox{0.292}{0.667}{\paperwidth}{\fontsize{15}{0}\fontcelestiaantiqua\scshape olympia}
	\placetextbox{0.357}{0.600}{\paperwidth}{\fontsize{15}{0}\fontcelestiaantiqua\scshape ilysium}
	\placetextbox{0.595}{0.714}{\paperwidth}{\fontsize{12}{0}\fontcelestiaantiqua\scshape nachtdock}
	\placetextbox{0.640}{0.657}{\paperwidth}{\fontsize{15}{0}\fontcelestiaantiqua\scshape ormos}
	\placetextbox{0.714}{0.839}{\paperwidth}{\fontsize{15}{0}\fontcelestiaantiqua\scshape die verbliebenen}
	
	\placetextbox{0.089}{0.462}{0.4\paperwidth}{\fontsize{9}{0}\fontcelestiaantiqua
{\textsc{\fontsize{14}{0}\fontcelestiaantiqua \phantom{.....}  Treiben im ewigen Blau}} \vspace{2pt} \\
\phantom{.....} Die Welten der Bläue schweben in einem Himmel aus sauerstoffreichem Gas und umkreisen dabei einen kleinen, kalten Stern. Gelehrte postulieren, dass dieser Stern aus reiner Essenz besteht—die seltsame Kraft, die Beschwörern erlaubt, Magie zu wirken. Das \enquote{Sonnensystem} ist kleiner, als man es vielleicht erwarten würde—in etwa sechs Wochen kann man mit einem üblichen Luftschiff von einem Ende das andere erreichen. Ein Großteil des Imperiums liegt so nah zusammen, dass man innerhalb von einem oder zwei Tagen die anderen Welten erreichen kann. \vspace{5pt} \\
{\fontsize{14}{0} \textsc{Die unteren Tiefen}} \vspace{2pt} \\
Unter dem \enquote{Himmel} der Bläue bilden schwerere Gase einen dichten Nebel. Diese Gase sind ätzend—zum Atmen benötigt man Gasmasken und die Mäntel von Luftschiffen fangen bereits nach kurzer Zeit an, sich zu zersetzen. Piraten und andere Verbrecher versuchen manchmal, die Tiefen zu nutzen um imperialen Truppen zu entkommen oder Überraschungsangriffe zu starten. Doch selbst wenn man vor den Gasen geschützt ist, leben dort Sternen-Kalmare und andere Monstrositäten... \vspace{5pt} \\
{\fontsize{14}{0} \textsc{Namen}} \vspace{2pt} \\
\textsc{Vornamen (M)}: Abel, Artemis, August, Eli, Giovanni, Ivan, Jack, Jefferson, Jonas,
Leo, Logan, Malachi, Mario, Micah, Nahum, Noah, Orlence, Oscar,
Samuel, Silas, Victor, Vlad, Wester \vspace{1pt} \\
\textsc{Vornamen (W)}: Alice, Ardent, Ashlyn, Caess, Clare, Elena, Eveline, Fiona,
Grace, Hannah, Hazel, Hester, Isabel, Krista, Jezebel, Leah, Lucile,
Lydia, Seraphina, Sonya, Sophie, Veronica, Violet \vspace{1pt} \\
\textsc{Nachnamen}: Bell, Bowen, Canter, Carson, Cross, Harwood, Hollas,
Hunter, Kalra, Keel, Moreau, Morgan, Porter, Pickett, Quinn, Sidhu,
Soto, Torrez, Vakharia, Walker, Winter, Wright \vspace{1pt} \\
\phantom{.....} \textsc{Adelshäuser}: Ash, Blackbird, Firefly, Mooncloud, 
\linebreak \phantom{.........} Nightsong, Snow, Twilight, Whitethorn.	
}
	\placetextbox{0.505}{0.462}{0.415\paperwidth}{\fontsize{9}{0}\fontcelestiaantiqua
{\fontsize{14}{0} \textsc{Ilysium}} \vspace{2pt} \\
Die Welt der Hauptstadt des Imperiums und Heimat für \\
viele der Adelshäuser. Das Leben auf Ilysium ist reich und dekadent und wird geprägt von Dienern, Sklaven und den speziell ausgebildeten Leibwachen des Adels. \vspace{3pt} \\
{\fontsize{14}{0} \textsc{Olympia}} \vspace{2pt} \\
Die Hauptstation der imperialen Luftflotte. Ausgehend von Olympia werden Expeditionen überall in die Bläue koordiniert, insbesondere zur Kolonialisierung. Außerdem lassen sich hier die besten Brauereien und Schnapsbrenner der imperialen Welten finden. \vspace{3pt} \\
{\fontsize{14}{0} \textsc{Ormos}} \vspace{2pt} \\
Die bekannteste der freien Welten. In den sich immer weiter ausbreitenden Städten versucht die Handelsunion Gesetz und Ordnung unter den Clans und Fraktionen der Freivölker zu halten. Sklaverei ist auf Ormos verboten, weshalb viele Ex-Sklaven die Welt als ihre neue Heimat wählen. \vspace{3pt} \\
{\fontsize{14}{0} \textsc{Nachtdock}} \vspace{2pt} \\
Nachtdock ist der wenigen Welten der Bläue, die nicht rotiert, sodass eine Seite dauerhaft in der Dunkelheitliegt. Auf dieser Seite existiert eine von Piraten und Schmugglern errichtete Hafenstadt. Dieser ruchlose Ort ist für seine illegalen Geschäfte bekannt, sowie für die Gefahren der herumstreunenden Diebe und Mörderer. Man sagt, alles was es gibt kann hier gekauft und verkauft werden, inklusive von Geheimnissen. \vspace{3pt} \\
{\fontsize{14}{0} \textsc{Die Verbliebenen}} \vspace{2pt} \\
Ein riesiger Strudel von gebrochenen Welten-Splittern. Die Verbliebenen sind sogar für die besten Piloten eine Unmöglichkeit zu durchfliegen. Es gibt Gerüchte, dass Uriah Flint, der König der Piraten, seine geheime Festung irgendwo in diesem Strudel \\
versteckt hält und nur die, die den geheimen Weg dorthin \\
kennen, können sie erreichen.
}